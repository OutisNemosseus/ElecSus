\documentclass[11pt,a4paper]{article}
\usepackage[utf8]{inputenc}
\usepackage{amsmath,amssymb,amsthm}
\usepackage{physics}
\usepackage{listings}
\usepackage{xcolor}
\usepackage{geometry}
\usepackage{hyperref}
\usepackage{booktabs}

\geometry{margin=1in}

\lstset{
    language=Python,
    basicstyle=\ttfamily\small,
    keywordstyle=\color{blue},
    commentstyle=\color{green!60!black},
    stringstyle=\color{orange},
    numbers=left,
    numberstyle=\tiny\color{gray},
    breaklines=true,
    frame=single,
    backgroundcolor=\color{gray!10}
}

\newtheorem{axiom}{Axiom}
\newtheorem{definition}{Definition}
\newtheorem{theorem}{Theorem}
\newtheorem{lemma}{Lemma}
\newtheorem{corollary}{Corollary}

\title{\textbf{ang\_mom.py} \\ Angular Momentum Matrices \\ \large Cartesian Components $J_x$, $J_y$, $J_z$}
\author{ElecSus Library Documentation}
\date{}

\begin{document}

\maketitle

\begin{abstract}
This document provides comprehensive documentation for \texttt{ang\_mom.py}, which constructs the matrix representations of angular momentum operators in Cartesian coordinates. These matrices are essential for computing fine structure, hyperfine structure, and Zeeman interactions in atomic physics calculations.
\end{abstract}

\tableofcontents
\newpage

%==============================================================================
\section{Theoretical Foundation}
%==============================================================================

\subsection{Angular Momentum in Quantum Mechanics}

\begin{axiom}[Hermiticity of Observables]
Physical observables correspond to Hermitian operators. The angular momentum components $J_x$, $J_y$, $J_z$ satisfy:
\begin{equation}
J_i^\dagger = J_i
\end{equation}
\end{axiom}

\begin{theorem}[Ladder Operator Decomposition]
The Cartesian angular momentum components can be expressed in terms of ladder operators:
\begin{align}
J_x &= \frac{1}{2}(J_+ + J_-) \\
J_y &= \frac{1}{2i}(J_+ - J_-) = -\frac{i}{2}(J_+ - J_-)\\
J_z &= \frac{1}{2}[J_+, J_-]_- = \frac{1}{2}(J_+ J_- - J_- J_+)
\end{align}
where $J_\pm = J_x \pm iJ_y$.
\end{theorem}

\begin{proof}
From $J_+ = J_x + iJ_y$ and $J_- = J_x - iJ_y$:
\begin{align}
J_+ + J_- &= 2J_x \implies J_x = \frac{1}{2}(J_+ + J_-) \\
J_+ - J_- &= 2iJ_y \implies J_y = \frac{1}{2i}(J_+ - J_-)
\end{align}
For $J_z$, use $[J_+, J_-] = 2\hbar J_z$ (in units $\hbar = 1$).
\end{proof}

\begin{definition}[Adjoint Relationship]
The lowering operator is the Hermitian adjoint of the raising operator:
\begin{equation}
J_- = J_+^\dagger
\end{equation}
For real matrix representations, this reduces to the transpose: $J_- = J_+^T$.
\end{definition}

\begin{corollary}[Hermiticity Verification]
Using $J_- = J_+^\dagger$:
\begin{align}
J_x^\dagger &= \frac{1}{2}(J_+^\dagger + J_-^\dagger) = \frac{1}{2}(J_- + J_+) = J_x \\
J_y^\dagger &= \frac{i}{2}(J_+^\dagger - J_-^\dagger) = \frac{i}{2}(J_- - J_+) = J_y
\end{align}
confirming $J_x$ and $J_y$ are Hermitian.
\end{corollary}

%==============================================================================
\section{Line-by-Line Code Analysis}
%==============================================================================

\subsection{Module Imports}

\begin{lstlisting}
from numpy import transpose,dot
import ang_mom_p
\end{lstlisting}
\textit{Import NumPy's transpose and dot product, plus the ladder operator module.}

\subsection{The $J_x$ Function}

\begin{lstlisting}
def jx(jj):
    jp=ang_mom_p.jp(jj)
    jm=transpose(jp)
    jx=0.5*(jp+jm)
    return jx
\end{lstlisting}
\begin{equation}
J_x = \frac{1}{2}(J_+ + J_-)
\end{equation}
\textit{Construct $J_x$ from the symmetric combination of ladder operators. The transpose gives $J_- = J_+^T$ since $J_+$ is real.}

\subsection{The $J_y$ Function}

\begin{lstlisting}
def jy(jj):
    jp=ang_mom_p.jp(jj)
    jm=transpose(jp)
    jy=0.5j*(jm-jp)
    return jy
\end{lstlisting}
\begin{equation}
J_y = \frac{i}{2}(J_- - J_+) = -\frac{i}{2}(J_+ - J_-)
\end{equation}
\textit{The antisymmetric combination with factor $i/2$ yields a Hermitian matrix. Note the sign convention: $J_y = 0.5j \cdot (J_- - J_+)$.}

\subsection{The $J_z$ Function}

\begin{lstlisting}
def jz(jj):
    jp=ang_mom_p.jp(jj)
    jm=transpose(jp)
    jz=0.5*(dot(jp,jm)-dot(jm,jp))
    return jz
\end{lstlisting}
\begin{equation}
J_z = \frac{1}{2}(J_+ J_- - J_- J_+) = \frac{1}{2}[J_+, J_-]
\end{equation}
\textit{The commutator of ladder operators gives the $z$-component. This yields a diagonal matrix with eigenvalues $m = j, j-1, \ldots, -j$.}

%==============================================================================
\section{Mathematical Properties}
%==============================================================================

\subsection{Matrix Structure}

\begin{theorem}[Matrix Forms]
For angular momentum $j$:
\begin{itemize}
\item $J_z$: Diagonal with entries $\{j, j-1, \ldots, -j\}$
\item $J_x$: Real symmetric, tri-diagonal
\item $J_y$: Purely imaginary antisymmetric (Hermitian), tri-diagonal
\end{itemize}
\end{theorem}

\subsection{Spin-1/2 Example}

For $j = 1/2$, the Pauli matrix representation (in units $\hbar = 1$):
\begin{equation}
J_x = \frac{1}{2}\sigma_x = \frac{1}{2}\begin{pmatrix} 0 & 1 \\ 1 & 0 \end{pmatrix}
\end{equation}
\begin{equation}
J_y = \frac{1}{2}\sigma_y = \frac{1}{2}\begin{pmatrix} 0 & -i \\ i & 0 \end{pmatrix}
\end{equation}
\begin{equation}
J_z = \frac{1}{2}\sigma_z = \frac{1}{2}\begin{pmatrix} 1 & 0 \\ 0 & -1 \end{pmatrix}
\end{equation}

\subsection{Verification of Commutation Relations}

The constructed matrices satisfy:
\begin{equation}
[J_x, J_y] = iJ_z, \quad [J_y, J_z] = iJ_x, \quad [J_z, J_x] = iJ_y
\end{equation}
(in units where $\hbar = 1$).

%==============================================================================
\section{Applications in ElecSus}
%==============================================================================

\subsection{Fine Structure Interaction}

The spin-orbit coupling uses:
\begin{equation}
H_{FS} = A_{FS} \vec{L} \cdot \vec{S} = A_{FS}(L_x S_x + L_y S_y + L_z S_z)
\end{equation}

\subsection{Hyperfine Structure}

The magnetic dipole hyperfine interaction:
\begin{equation}
H_{HFS} = A_{HFS} \vec{I} \cdot \vec{J} = A_{HFS}(I_x J_x + I_y J_y + I_z J_z)
\end{equation}

\subsection{Zeeman Interaction}

The magnetic field coupling:
\begin{equation}
H_Z = \mu_B B_z (g_L L_z + g_S S_z + g_I I_z)
\end{equation}
uses the $z$-components directly for fields along $\hat{z}$.

%==============================================================================
\section{Numerical Verification}
%==============================================================================

\subsection{Total Angular Momentum Squared}

The identity $J^2 = j(j+1)\mathbb{1}$ can verify correctness:
\begin{equation}
J^2 = J_x^2 + J_y^2 + J_z^2 = j(j+1) \mathbb{1}_{(2j+1)\times(2j+1)}
\end{equation}

\subsection{Eigenvalue Check}

$J_z$ should have eigenvalues $\{j, j-1, \ldots, -j\}$ appearing on the diagonal in descending order.

%==============================================================================
\section{Summary}
%==============================================================================

The \texttt{ang\_mom.py} module provides:

\begin{enumerate}
\item \texttt{jx(j)}: Returns $(2j+1) \times (2j+1)$ matrix for $J_x$
\item \texttt{jy(j)}: Returns $(2j+1) \times (2j+1)$ matrix for $J_y$  
\item \texttt{jz(j)}: Returns $(2j+1) \times (2j+1)$ matrix for $J_z$
\end{enumerate}

Key features:
\begin{itemize}
\item Uses ladder operator construction for numerical stability
\item Works for any half-integer or integer $j$
\item Matrices satisfy $\text{SU}(2)$ algebra
\item Called by \texttt{fs\_hfs.py} and \texttt{sz\_lsi.py} for building atomic Hamiltonians
\end{itemize}

\end{document}
